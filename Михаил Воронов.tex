\documentclass[11pt]{article}
\usepackage[T2A]{fontenc}
\usepackage[utf8]{inputenc}
\usepackage[english, russian]{babel}
\usepackage{qtree}
\usepackage{avm}
\usepackage{indentfirst}


%Gummi|065|=)
\title{\textbf{Домашнее задание по ЛФГ}}
\author{Михаил Воронов}
\date{}
\begin{document}

\maketitle

\section{}

\subsection{Предложение (a)}
В этом предложении нет ничего страшного.

\Tree
[
	.S
	[
		.NP\\{\tiny($\uparrow$GF)=$\downarrow$}
		[
			.N\\{\tiny$\uparrow$=$\downarrow$} Зауыр
		]
	]
	[
		.VP\\{\tiny$\uparrow$=$\downarrow$}
		[
			[
				.NP\\{\tiny($\uparrow$GF)=$\downarrow$}\\{\tiny($\uparrow$FOC)=$\downarrow$}
				[
					.N\\{\tiny$\uparrow$=$\downarrow$} [ Аланы ]
				]
			]
			[
				.V'
				[
					.V\\{\tiny$\uparrow$=$\downarrow$} [ фед-та ]
				]
			]
		]
	]
]

\begin{avm}
\[
	PRED = & \rm ‘увидеть\q<SUBJ, OBJ\q>’\\
	TAM = & $past$\\
	SUBJ = & \[
		PRED  = & `Заур' \\
		CASE = & $nom$
	\]\\
	OBJ = & \[
		PRED = & `Алан' \\
		CASE = & $gen$
	\]
\]
\end{avm}

\subsection{Предложение (b)}
Структуры для этого предложения уже покажутся чуть более контринтуитивными. Однако, они соответствуют правилам. Интересно отметить, что f-структура совпадает с первым предложением. Я в точности не помню, как в ЛФГ обходят эту проблему, кажется, просто обводят кружочком то, что в фокусе, пишут отдельно слово FOC и соединяют их. В первом предложении тогда обведён будет SUBJ, во втором объведён будет OBJ.

Однако, я всё равно не умею так рисовать в техе, поэтому здесь и далее (и до) этого не будет.

\Tree
[
	.S
	[
		.NP\\{\tiny($\uparrow$GF)=$\downarrow$}
		[
			.N\\{\tiny$\uparrow$=$\downarrow$} Аланы
		]
	]
	[
		.VP\\{\tiny$\uparrow$=$\downarrow$}
		[
			[
				.NP\\{\tiny($\uparrow$GF)=$\downarrow$}\\{\tiny($\uparrow$FOC)=$\downarrow$}
				[
					.N\\{\tiny$\uparrow$=$\downarrow$}
						[ Зауыр ]
				]
			]
			[
				.V'
				[
					.V\\{\tiny$\uparrow$=$\downarrow$}
						[ фед-та ]
				]
			]
		]
	]
]


\begin{avm}
\[
	PRED = & \rm ‘увидеть\q<SUBJ, OBJ\q>’\\
	TAM = & $past$\\
	SUBJ = & \[
		PRED  = & `Заур' \\
		CASE  = & $nom$
	\]\\
	OBJ = & \[
		PRED = & `Алан' \\
		CASE = & $gen$
	\]
\]
\end{avm}

\subsection{Предложение (c)}
Тоже довольно очевидное предложение без подводных камней.

\Tree
[
	.S
	[
		.NP\\{\tiny($\uparrow$GF)=$\downarrow$}
		[
			.N\\{\tiny$\uparrow$=$\downarrow$} Зауыр
		]
	]
	[
		.VP\\{\tiny$\uparrow$=$\downarrow$}
		[
			[
				.V'
				[
					.V\\{\tiny$\uparrow$=$\downarrow$} [ эрбацыди ]
				]
			]
		]
	]
]

\begin{avm}
\[
	PRED = & \rm ‘придти\q<SUBJ\q>’\\
	TAM = & $past$\\
	SUBJ = & \[
		PRED = & `Заур' \\
		CASE = & $nom$
	\]\\
\]
\end{avm}


\subsection{Предложение (d)}
Это предложение невозможно из-за противоречия TAM в группе комплементайзера: комплементайзер требует субъюнктив, а глагол стоит в форме прошедшего времени. Это предложение можно исправить, изменив форму глагола на субъюнктив.

\begin{avm}
\[
	PRED = & \rm ‘хотеть\q<SUBJ, COMP\q>’\\
	TAM = & $prs$\\
	SUBJ = & \[
		PRED = & `Заур' \\
		CASE = & $gen$
	\]\\
	COMP = & \[
		CTYPE = & $purp$\\
		PRED = & \rm ‘придти\q<SUBJ\q>’\\
		TAM = & $past$ & $\leftarrow$ошибка здесь, должен быть $sbjv$\\
		SUBJ = & \[
			PRED = & 'Алан'\\
			CASE = & $nom$
		\]
	\]
\]
\end{avm}


\subsection{Предложение (e)}
Это предложение возможно.

\Tree
[
	.S
	[
		.NP\\{\tiny($\uparrow$GF)=$\downarrow$}
		[
			.N\\{\tiny$\uparrow$=$\downarrow$} Аланы
		]
	]
	[
		.VP\\{\tiny$\uparrow$=$\downarrow$}
		[
			[ .V'
				[
					.V\\{\tiny$\uparrow$=$\downarrow$} [ фэдны ]
				]
				[
					.CP\\{\tiny($\uparrow$COMP)=$\downarrow$}
					[
						.NP\\{\tiny($\uparrow$GF)=$\downarrow$}
						[
							.NP\\{\tiny($\uparrow$POSS)=$\downarrow$}\\{\tiny($\downarrow$CASE)=(c)gen}
							[
								.N\\{\tiny$\uparrow$=$\downarrow$} Зауыры
							]
						]
						[
							.N\\{\tiny$\uparrow$=$\downarrow$} эфсымэр
						]
					]
					[
						.C'
						[
							.C\\{\tiny$\uparrow$=$\downarrow$} цэмэй
						]
						[
							.S
							[
								.VP
								[
									.V'
									[
										.V\\{\tiny$\uparrow$=$\downarrow$} эрбацэуа
									]
								]
							]
						]
					]
				]
			]
		]
	]
] 

\begin{avm}
\[
	PRED = & \rm ‘хотеть\q<SUBJ, COMP\q>’\\
	TAM = & $prs$\\
		SUBJ = & \[
		PRED = & `Алан' \\
		CASE = & $gen$
	\]\\
	COMP = & \[
		CTYPE = & $purp$\\
		PRED = & \rm ‘придти\q<SUBJ\q>’\\
		TAM = & $sbjv$\\
		SUBJ = & \[
			PRED = & 'брат'\\
			CASE = & $nom$\\
			POSS = & \[
				PRED = & 'Заур'\\
				CASE = & $gen$
			\]
		\]
	\]
\]
\end{avm}

\subsection{Предложение (f)}
Невозможно из-за конфликта с правилом (2). Мы не можем запихнать ничего похожего на VP перед комплементайзером. Исправленный вариант ниже.

\Tree
[
	.S
	[
		.NP\\{\tiny($\uparrow$GF)=$\downarrow$}
		[
			.NP\\{\tiny($\uparrow$POSS)=$\downarrow$}\\{\tiny($\downarrow$CASE)=(c)gen}
			[ .N Зауыры ]
		]
		[
			.N\\{\tiny$\uparrow$=$\downarrow$} эфсымэры
		]
	]
	[
		.VP\\{\tiny$\uparrow$=$\downarrow$}
		[
			[
				.V'
				[
					.V\\{\tiny$\uparrow$=$\downarrow$} [ фэдны ]
				]
				[
					.CP\\{\tiny($\uparrow$COMP)=$\downarrow$}
					[
						.NP\\{\tiny($\uparrow$GF)=$\downarrow$}
							[
								.N\\{\tiny$\uparrow$=$\downarrow$} Алан
							]
					]
					[
						.C'
						[
							.C\\{\tiny$\uparrow$=$\downarrow$} цэмэй
						]
						[
							.S
							[
								.VP
								[
									.V'
									[
										.V\\{\tiny$\uparrow$=$\downarrow$} эрбацэуа
									]
								]
							]
						]
					]
				]
			]
		]
	]
] 

\begin{avm}
\[
	PRED = & \rm ‘хотеть\q<SUBJ, COMP\q>’\\
	TAM = & $prs$\\
	SUBJ = & \[
		PRED = & `брат' \\
		CASE = & $gen$\\
		POSS = & \[
			PRED = & 'Заур'\\
			CASE = & $gen$
		\]
	\]\\
	COMP = & \[
		CTYPE = & $purp$\\
		PRED = & \rm ‘придти\q<SUBJ\q>’\\
		TAM = & $sbjv$ \\
		SUBJ = & \[
			PRED = & 'Алан'\\
			CASE = & $nom$
		\]
	\]
\]
\end{avm}

\subsection{Предложение (g)}
Невозможно по правилу (1), так как "чи" требует фокус, а S его не принимает.
Исправленный вариант: Аланы чи федта?

\Tree
[
	.S
	[
		.NP\\{\tiny($\uparrow$GF)=$\downarrow$}
		[
			.N\\{\tiny$\uparrow$=$\downarrow$} Аланы
		]
	]
	[
		.VP\\{\tiny$\uparrow$=$\downarrow$}
		[
			[
				.NP\\{\tiny($\uparrow$GF)=$\downarrow$}\\{\tiny($\uparrow$FOC)=$\downarrow$}
				[
					.N\\{\tiny$\uparrow$=$\downarrow$} [ чи ]
				]
			]
			[
				.V'
				[
					.V\\{\tiny$\uparrow$=$\downarrow$} [ фед-та ]
				]
			]
		]
	]
]

\begin{avm}
\[
	PRED = & \rm ‘увидеть\q<SUBJ, OBJ\q>’\\
	TAM = & $past$\\
	SUBJ = & \[
		PRED  = & `кто' \\
		CASE = & $nom$ \\
		FOC = & true
	\]\\
	OBJ = & \[
		PRED = & `Алан' \\
		CASE = & $gen$
	\]
\]
\end{avm}

\subsection{Предложение (h)}
Никаких проблем.

\Tree
[
	.S
	[
		.NP\\{\tiny($\uparrow$GF)=$\downarrow$}
		[
			.N\\{\tiny$\uparrow$=$\downarrow$} Алан
		]
	]
	[
		.VP\\{\tiny$\uparrow$=$\downarrow$}
		[
			[
				.NP\\{\tiny($\uparrow$GF)=$\downarrow$}\\{\tiny($\uparrow$FOC)=$\downarrow$}
				[
					.N\\{\tiny$\uparrow$=$\downarrow$} [ кэй ]
				]
			]
			[
				.V'
				[
					.V\\{\tiny$\uparrow$=$\downarrow$} [ фед-та ]
				]
			]
		]
	]
]

\begin{avm}
\[
	PRED = & \rm ‘увидеть\q<SUBJ, OBJ\q>’\\
	TAM = & $past$\\
	SUBJ = & \[
		PRED  = & `Алан' \\
		CASE = & $nom$
	\]\\
	OBJ = & \[
		PRED = & `кто' \\
		CASE = & $gen$ \\
		FOC = & true
	\]
\]
\end{avm}


\section{}
Самое очевидное, что тут можно почерпнуть, это то, что порядок SOV (правила 3 и 4). Однако, как и во многих языках с порядком слов SOV, здесь встречается порядок SVO в случае, когда в качестве прямого дополнения выступает клауза с комплементайзером (правила 3, 4 и 5).

Это, вероятно, связано с тяжестью прямого дополнение (именную группу можно разместить между, а целую клаузу уже неудобно).

Так же возможно вынесение субъекта во вторую позицию, если он в фокусе.

В итоге у нас получается базовый порядок слов SOV (SVO, если объект комплементайзер, и OSV, если субъект в фокусе). Для клаузы с комплементайзером базовым порядком слов будут CSOV и SCOV.

Так же стоит отметить, что язык левоветвящий: посессоры в генитиве (правило 6) расположены слева.

\end{document}
